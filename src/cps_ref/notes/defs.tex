\usepackage{fullpage}
\usepackage{amsmath}
\usepackage{nccmath}
\usepackage{minted}
\usepackage{semantic}
\usepackage{xparse}
\usepackage{fancyvrb}
\usepackage{lmodern}
\usepackage{xspace}
\usepackage{mathpartir}
\usepackage{stmaryrd}
\usepackage{booktabs}
\usepackage{mathtools}
\usepackage{amssymb}
\usepackage{listings}
\usepackage{xcolor}

\makeatletter
\let\save@mathaccent\mathaccent
\newcommand*\if@single[3]{%
  \setbox0\hbox{${\mathaccent"0362{#1}}^H$}%
  \setbox2\hbox{${\mathaccent"0362{\kern0pt#1}}^H$}%
  \ifdim\ht0=\ht2 #3\else #2\fi
  }
%The bar will be moved to the right by a half of \macc@kerna, which is computed by amsmath:
\newcommand*\rel@kern[1]{\kern#1\dimexpr\macc@kerna}
%If there's a superscript following the bar, then no negative kern may follow the bar;
%an additional {} makes sure that the superscript is high enough in this case:
\newcommand*\widebar[1]{\@ifnextchar^{{\wide@bar{#1}{0}}}{\wide@bar{#1}{1}}}
%Use a separate algorithm for single symbols:
\newcommand*\wide@bar[2]{\if@single{#1}{\wide@bar@{#1}{#2}{1}}{\wide@bar@{#1}{#2}{2}}}
\newcommand*\wide@bar@[3]{%
  \begingroup
  \def\mathaccent##1##2{%
%Enable nesting of accents:
    \let\mathaccent\save@mathaccent
%If there's more than a single symbol, use the first character instead (see below):
    \if#32 \let\macc@nucleus\first@char \fi
%Determine the italic correction:
    \setbox\z@\hbox{$\macc@style{\macc@nucleus}_{}$}%
    \setbox\tw@\hbox{$\macc@style{\macc@nucleus}{}_{}$}%
    \dimen@\wd\tw@
    \advance\dimen@-\wd\z@
%Now \dimen@ is the italic correction of the symbol.
    \divide\dimen@ 3
    \@tempdima\wd\tw@
    \advance\@tempdima-\scriptspace
%Now \@tempdima is the width of the symbol.
    \divide\@tempdima 10
    \advance\dimen@-\@tempdima
%Now \dimen@ = (italic correction / 3) - (Breite / 10)
    \ifdim\dimen@>\z@ \dimen@0pt\fi
%The bar will be shortened in the case \dimen@<0 !
    \rel@kern{0.6}\kern-\dimen@
    \if#31
      \overline{\rel@kern{-0.6}\kern\dimen@\macc@nucleus\rel@kern{0.4}\kern\dimen@}%
      \advance\dimen@0.4\dimexpr\macc@kerna
%Place the combined final kern (-\dimen@) if it is >0 or if a superscript follows:
      \let\final@kern#2%
      \ifdim\dimen@<\z@ \let\final@kern1\fi
      \if\final@kern1 \kern-\dimen@\fi
    \else
      \overline{\rel@kern{-0.6}\kern\dimen@#1}%
    \fi
  }%
  \macc@depth\@ne
  \let\math@bgroup\@empty \let\math@egroup\macc@set@skewchar
  \mathsurround\z@ \frozen@everymath{\mathgroup\macc@group\relax}%
  \macc@set@skewchar\relax
  \let\mathaccentV\macc@nested@a
%The following initialises \macc@kerna and calls \mathaccent:
  \if#31
    \macc@nested@a\relax111{#1}%
  \else
%If the argument consists of more than one symbol, and if the first token is
%a letter, use that letter for the computations:
    \def\gobble@till@marker##1\endmarker{}%
    \futurelet\first@char\gobble@till@marker#1\endmarker
    \ifcat\noexpand\first@char A\else
      \def\first@char{}%
    \fi
    \macc@nested@a\relax111{\first@char}%
  \fi
  \endgroup
}

\renewcommand{\ttdefault}{pcr}

\lstdefinelanguage{lambdalr}{
  basicstyle=\ttfamily\small,
  morekeywords={fn,ret,let,letcont,if,then,else,jump,own,new,call},
  basewidth=0.6em,
  mathescape
}
\lstnewenvironment{lambdalr}{\lstset{language=lambdalr}}{}

\usemintedstyle[lambda_lr_cps.py -x]{bw}


\newcommand\@jointhree[4][;]{#2#1#3#1#4}
\newcommand\@jointwo[3][;]{#2#1#3}


% EXPRESSIONS
\newcommand\kw[1]{\mathbf{#1}} % keywords

\newcommand\letvar[2]{\kw{let}\;#1=#2}
\NewDocumentCommand\letcont{m>{\SplitArgument{1}{;}}r[]mmmm}
  {\kw{letcont}\;#1[\@jointwo#2](\bar{#3}.#4) = #5\;\kw{in}\;#6}
\newcommand\ite[3]{\kw{if}\;{#1}\;\kw{then}\;#2\;\kw{else}\;#3}
\newcommand\call[4]{\kw{call}\;#1[#2](#3)\;\kw{ret}\;#4}
\newcommand\assign[2]{#1 \coloneqq #2}
\newcommand\jump[1]{\kw{jump}\; #1}
\newcommand\new[1]{\kw{new}(#1)}
\newcommand\abort{\kw{abort}\xspace}

\reservestyle{\keyword}{\mathbf}
\keyword{true,false}
\reservestyle{\basictype}{\mathtt}
\basictype{int,bool}
\reservestyle{\syntaxset}{\textit}
\syntaxset{Place,Path,Const,Op,RVal,FuncBody,BType,Type,Pred,TEnv,KEnv,Constraint,Instr,GEnv,Subst}


% TYPES
\NewDocumentCommand\cont{>{\SplitArgument{1}{;}}r[]>{\SplitArgument{1}{;}}m}
  {\kw{cont}[\@jointwo#1](\@jointwo#2)}
\NewDocumentCommand\fn
  {>{\SplitArgument{2}{;}}m>{\SplitArgument{1}{;}}m}
  {\kw{fn}\@fn#1 \to \@jointwo[/]#2}
\newcommand\@fn[3]{[#1;#2](#3)}
\newcommand{\reft}[2]{\ensuremath{\{#1\mid#2\}}}
\RenewDocumentCommand\r{>{\SplitArgument{1}{|}}m}{\reft#1}
\newcommand{\uninit}{\lightning}
\newcommand{\own}[1]{\text{\ttfamily \bfseries own}_{#1}}


%% JUDGEMENTS
\definecolor{light-gray}{gray}{0.9}
\newcommand{\iscopy}[1]{#1~\text{copy}}
\newcommand{\isnoncopy}[1]{#1~\text{noncopy}}
\newcommand{\tread}[6]{#1;#2\vdash #3 \rightsquigarrow #4 \dashv #5;#6}
\newcommand{\trval}[6]{#1;#2\vdash \colorbox{light-gray}{$#3$}~: #4 \rightsquigarrow #5;#6}
\newcommand{\tinstr}[5]{#1;#2\vdash \colorbox{light-gray}{$#3$} \rightsquigarrow #4;#5}
\newcommand{\tfun}[4]{#1;#2|#3\vdash \colorbox{light-gray}{$#4$}}
\newcommand{\envincl}[2]{#1 \Rightarrow #2}
\newcommand{\lookup}[3]{(#1;#2)(#3)}
\newcommand{\venv}{\mathbf{T}}  % Value context
\newcommand{\cenv}{\mathbf{C}}  % Constraints
\newcommand{\lenv}{\Delta}      % Location context
\newcommand{\kenv}{\mathbf{K}}  % Continuation context


%% METAFUNCTIONS
\NewDocumentCommand\envsubst{>{\SplitArgument{1}{;}}m>{\SplitArgument{1}{;}}mm}
  {\@jointwo#1 \Rightarrow^{#3} \@jointwo#2}
\NewDocumentCommand\typsubst{>{\SplitArgument{1}{;}}m>{\SplitArgument{1}{;}}mm}
  {\@jointwo#1 \Rightarrow^{#3} \@jointwo#2}
\NewDocumentCommand\subst{>{\SplitArgument{1}{;}}m}{\mathtt{subst}(\@jointwo#1)}
\NewDocumentCommand\self{>{\SplitArgument{1}{;}}m}{\mathtt{self}(\@jointwo[,]#1)}
\newcommand{\sizeof}[1]{\mathtt{size}(#1)}
\newcommand\extract[1]{\llbracket #1 \rrbracket}

\newcommand{\@where}{\text{\ttfamily\bfseries where}}
\newenvironment{fleqnarray}[2][1em]
  {\begin{fleqn}[#1]\begin{displaymath}\begin{array}{#2}}
  {\end{array}\end{displaymath}\end{fleqn}}
\newenvironment{metafun}[1]{\noindent$\boxed{#1}$\begin{fleqnarray}{lcl}}{\end{fleqnarray}}
\NewExpandableDocumentCommand\where{sm}{
  \multicolumn{3}{l}{\quad\IfBooleanTF{#1}{\phantom{\@where}}{\@where}\;\;#2}
  \\
}
\NewExpandableDocumentCommand\whereblock{t>sm}{
  \IfBooleanT{#1}{\quad\@where\\}
  \multicolumn{3}{l}{
    \arraycolsep=2pt
    \IfBooleanTF{#1}
    {\qquad}
    {\quad\IfBooleanTF{#2}{\phantom{\@where}}{\@where}\hspace{-2pt}\;\;}
    \begin{array}[t]{lcl}#3\end{array}
  }
  \\
}


%% MISC
\renewcommand{\bar}{\widebar}
\renewcommand{\k}{\kappa}
\providecommand{\leftsquigarrow}{%
  \mathrel{\mathpalette\reflect@squig\relax}%
}
\newcommand{\reflect@squig}[2]{%
  \reflectbox{$\m@th#1\rightsquigarrow$}%
}
\DeclareMathSymbol{:}{\mathpunct}{operators}{"3A}

\makeatother