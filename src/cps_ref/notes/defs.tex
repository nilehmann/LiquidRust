\usepackage{fullpage}
\usepackage{amsmath}
\usepackage{nccmath}
\usepackage{minted}
\usepackage{semantic}
\usepackage{xparse}
\usepackage{fancyvrb}
\usepackage{lmodern}
\usepackage{xspace}
\usepackage{mathpartir}
\usepackage{stmaryrd}
\usepackage{booktabs}
\usepackage{mathtools}
\usepackage{amssymb}
\usepackage{listings}
\usepackage{xcolor}

\makeatletter

\renewcommand{\ttdefault}{pcr}

\lstdefinelanguage{lambdalr}{
  basicstyle=\ttfamily\small,
  morekeywords={fn,ret,let,letcont,if,then,else,jump,own,new,call},
  basewidth=0.6em,
  mathescape
}
\lstnewenvironment{lambdalr}{\lstset{language=lambdalr}}{}

\usemintedstyle[lambda_lr_cps.py -x]{bw}


\newcommand\@jointhree[4][;]{#2#1#3#1#4}
\newcommand\@jointwo[3][;]{#2#1#3}


% EXPRESSIONS
\newcommand\kw[1]{\mathbf{#1}} % keywords

\newcommand\letvar[2]{\kw{let}\;#1=#2}
\NewDocumentCommand\letcont{r[]>{\SplitArgument{2}{;}}mmm}
  {\kw{letcont}\;#1(\@jointhree#2)=#3\;\kw{in}\;#4}
\newcommand\ite[3]{\kw{if}\;{#1}\;\kw{then}\;#2\;\kw{else}\;#3}
\newcommand\call[2]{\kw{call}\;#1\;\kw{ret}\;#2}
\newcommand\assign[2]{#1 \coloneqq #2}
\newcommand\jump[1]{\kw{jump}\; #1}
\newcommand\new[1]{\kw{new}(#1)}
\newcommand\abort{\kw{abort}\xspace}

\reservestyle{\keyword}{\mathbf}
\keyword{true,false}
\reservestyle{\basictype}{\mathtt}
\basictype{int,bool}
\reservestyle{\syntaxset}{\textit}
\syntaxset{Place,Path,Const,Op,RVal,FuncBody,BType,Type,Pred,TEnv,KEnv,Constraint,Instr,GEnv,Subst}


% TYPES
\NewDocumentCommand\cont{>{\SplitArgument{2}{;}}m}{\kw{cont}(\@jointhree#1)}
\NewDocumentCommand\fn
  {>{\SplitArgument{1}{;}}m>{\SplitArgument{1}{;}}m}
  {\kw{fn}(\@jointwo#1) \to \@jointwo[/]#2}
\newcommand{\reft}[2]{\ensuremath{\{#1\mid#2\}}}
\RenewDocumentCommand\r{>{\SplitArgument{1}{|}}m}{\reft#1}
\newcommand{\uninit}{\lightning}
\newcommand{\own}[1]{\text{\ttfamily \bfseries own}\left(#1\right)}


%% JUDGEMENTS
\definecolor{light-gray}{gray}{0.9}
\newcommand{\iscopy}[1]{#1~\text{copy}}
\newcommand{\isnoncopy}[1]{#1~\text{noncopy}}
\newcommand{\tread}[6]{#1;#2\vdash #3 \rightsquigarrow #4 \dashv #5;#6}
\newcommand{\trval}[6]{#1;#2\vdash \colorbox{light-gray}{$#3$}~: #4 \rightsquigarrow #5;#6}
\newcommand{\tinstr}[5]{#1;#2\vdash \colorbox{light-gray}{$#3$} \rightsquigarrow #4;#5}
\newcommand{\tfun}[4]{#1;#2|#3\vdash \colorbox{light-gray}{$#4$}}
\newcommand{\envincl}[2]{#1 \Subset #2}
\newcommand{\lookup}[3]{(#1;#2)(#3)}
\newcommand{\tenv}{\mathbf{T}}
\newcommand{\kenv}{\mathbf{K}}


%% METAFUNCTIONS
\NewDocumentCommand\envsubst{>{\SplitArgument{1}{;}}m>{\SplitArgument{1}{;}}mm}
  {\@jointwo#1 \Rightarrow^{#3} \@jointwo#2}
\NewDocumentCommand\subst{>{\SplitArgument{1}{;}}m}{\mathtt{subst}(\@jointwo#1)}
\NewDocumentCommand\self{>{\SplitArgument{1}{;}}m}{\mathtt{self}(\@jointwo[,]#1)}
\newcommand{\sizeof}[1]{\mathtt{size}(#1)}
\newcommand\extract[1]{\llbracket #1 \rrbracket}

\newcommand{\@where}{\text{\ttfamily\bfseries where}}
\newenvironment{fleqnarray}[2][1em]
  {\begin{fleqn}[#1]\begin{displaymath}\begin{array}{#2}}
  {\end{array}\end{displaymath}\end{fleqn}}
\newenvironment{metafun}[1]{\noindent$\boxed{#1}$\begin{fleqnarray}{lcl}}{\end{fleqnarray}}
\NewExpandableDocumentCommand\where{sm}{
  \multicolumn{3}{l}{\quad\IfBooleanTF{#1}{\phantom{\@where}}{\@where}\;\;#2}
  \\
}
\NewExpandableDocumentCommand\whereblock{t>sm}{
  \IfBooleanT{#1}{\quad\@where\\}
  \multicolumn{3}{l}{
    \arraycolsep=2pt
    \IfBooleanTF{#1}
    {\qquad}
    {\quad\IfBooleanTF{#2}{\phantom{\@where}}{\@where}\hspace{-2pt}\;\;}
    \begin{array}[t]{lcl}#3\end{array}
  }
  \\
}


%% MISC
\renewcommand{\bar}{\overline}
\renewcommand{\k}{\kappa}
\providecommand{\leftsquigarrow}{%
  \mathrel{\mathpalette\reflect@squig\relax}%
}
\newcommand{\reflect@squig}[2]{%
  \reflectbox{$\m@th#1\rightsquigarrow$}%
}
\DeclareMathSymbol{:}{\mathpunct}{operators}{"3A}

\makeatother