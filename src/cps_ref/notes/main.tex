\documentclass{article}
\usepackage{fullpage}
\usepackage{amsmath}
\usepackage{nccmath}
\usepackage{minted}
\usepackage{semantic}
\usepackage{xparse}
\usepackage{fancyvrb}
\usepackage{lmodern}
\usepackage{xspace}
\usepackage{mathpartir}
\usepackage{stmaryrd}
\usepackage{booktabs}
\usepackage{mathtools}
\usepackage{amssymb}
\usepackage{listings}
\usepackage{xcolor}

\makeatletter
\let\save@mathaccent\mathaccent
\newcommand*\if@single[3]{%
  \setbox0\hbox{${\mathaccent"0362{#1}}^H$}%
  \setbox2\hbox{${\mathaccent"0362{\kern0pt#1}}^H$}%
  \ifdim\ht0=\ht2 #3\else #2\fi
  }
%The bar will be moved to the right by a half of \macc@kerna, which is computed by amsmath:
\newcommand*\rel@kern[1]{\kern#1\dimexpr\macc@kerna}
%If there's a superscript following the bar, then no negative kern may follow the bar;
%an additional {} makes sure that the superscript is high enough in this case:
\newcommand*\widebar[1]{\@ifnextchar^{{\wide@bar{#1}{0}}}{\wide@bar{#1}{1}}}
%Use a separate algorithm for single symbols:
\newcommand*\wide@bar[2]{\if@single{#1}{\wide@bar@{#1}{#2}{1}}{\wide@bar@{#1}{#2}{2}}}
\newcommand*\wide@bar@[3]{%
  \begingroup
  \def\mathaccent##1##2{%
%Enable nesting of accents:
    \let\mathaccent\save@mathaccent
%If there's more than a single symbol, use the first character instead (see below):
    \if#32 \let\macc@nucleus\first@char \fi
%Determine the italic correction:
    \setbox\z@\hbox{$\macc@style{\macc@nucleus}_{}$}%
    \setbox\tw@\hbox{$\macc@style{\macc@nucleus}{}_{}$}%
    \dimen@\wd\tw@
    \advance\dimen@-\wd\z@
%Now \dimen@ is the italic correction of the symbol.
    \divide\dimen@ 3
    \@tempdima\wd\tw@
    \advance\@tempdima-\scriptspace
%Now \@tempdima is the width of the symbol.
    \divide\@tempdima 10
    \advance\dimen@-\@tempdima
%Now \dimen@ = (italic correction / 3) - (Breite / 10)
    \ifdim\dimen@>\z@ \dimen@0pt\fi
%The bar will be shortened in the case \dimen@<0 !
    \rel@kern{0.6}\kern-\dimen@
    \if#31
      \overline{\rel@kern{-0.6}\kern\dimen@\macc@nucleus\rel@kern{0.4}\kern\dimen@}%
      \advance\dimen@0.4\dimexpr\macc@kerna
%Place the combined final kern (-\dimen@) if it is >0 or if a superscript follows:
      \let\final@kern#2%
      \ifdim\dimen@<\z@ \let\final@kern1\fi
      \if\final@kern1 \kern-\dimen@\fi
    \else
      \overline{\rel@kern{-0.6}\kern\dimen@#1}%
    \fi
  }%
  \macc@depth\@ne
  \let\math@bgroup\@empty \let\math@egroup\macc@set@skewchar
  \mathsurround\z@ \frozen@everymath{\mathgroup\macc@group\relax}%
  \macc@set@skewchar\relax
  \let\mathaccentV\macc@nested@a
%The following initialises \macc@kerna and calls \mathaccent:
  \if#31
    \macc@nested@a\relax111{#1}%
  \else
%If the argument consists of more than one symbol, and if the first token is
%a letter, use that letter for the computations:
    \def\gobble@till@marker##1\endmarker{}%
    \futurelet\first@char\gobble@till@marker#1\endmarker
    \ifcat\noexpand\first@char A\else
      \def\first@char{}%
    \fi
    \macc@nested@a\relax111{\first@char}%
  \fi
  \endgroup
}

\renewcommand{\ttdefault}{pcr}

\lstdefinelanguage{lambdalr}{
  basicstyle=\ttfamily\small,
  morekeywords={fn,ret,let,letcont,if,then,else,jump,own,new,call},
  basewidth=0.6em,
  mathescape
}
\lstnewenvironment{lambdalr}{\lstset{language=lambdalr}}{}

\usemintedstyle[lambda_lr_cps.py -x]{bw}


\newcommand\@jointhree[4][;]{#2#1#3#1#4}
\newcommand\@jointwo[3][;]{#2#1#3}


% EXPRESSIONS
\newcommand\kw[1]{\mathbf{#1}} % keywords

\newcommand\letvar[2]{\kw{let}\;#1=#2}
\NewDocumentCommand\letcont{m>{\SplitArgument{1}{;}}r[]mmmm}
  {\kw{letcont}\;#1[\@jointwo#2](\bar{#3}.#4) = #5\;\kw{in}\;#6}
\newcommand\ite[3]{\kw{if}\;{#1}\;\kw{then}\;#2\;\kw{else}\;#3}
\newcommand\call[4]{\kw{call}\;#1[#2](#3)\;\kw{ret}\;#4}
\newcommand\assign[2]{#1 \coloneqq #2}
\newcommand\jump[1]{\kw{jump}\; #1}
\newcommand\new[1]{\kw{new}(#1)}
\newcommand\abort{\kw{abort}\xspace}

\reservestyle{\keyword}{\mathbf}
\keyword{true,false}
\reservestyle{\basictype}{\mathtt}
\basictype{int,bool}
\reservestyle{\syntaxset}{\textit}
\syntaxset{Place,Path,Const,Op,RVal,FuncBody,BType,Type,Pred,TEnv,KEnv,Constraint,Instr,GEnv,Subst}


% TYPES
\NewDocumentCommand\cont{>{\SplitArgument{1}{;}}r[]>{\SplitArgument{1}{;}}m}
  {\kw{cont}[\@jointwo#1](\@jointwo#2)}
\NewDocumentCommand\fn
  {>{\SplitArgument{2}{;}}m>{\SplitArgument{1}{;}}m}
  {\kw{fn}\@fn#1 \to \@jointwo[/]#2}
\newcommand\@fn[3]{[#1;#2](#3)}
\newcommand{\reft}[2]{\ensuremath{\{#1\mid#2\}}}
\RenewDocumentCommand\r{>{\SplitArgument{1}{|}}m}{\reft#1}
\newcommand{\uninit}{\lightning}
\newcommand{\own}[1]{\text{\ttfamily \bfseries own}_{#1}}


%% JUDGEMENTS
\definecolor{light-gray}{gray}{0.9}
\newcommand{\iscopy}[1]{#1~\text{copy}}
\newcommand{\isnoncopy}[1]{#1~\text{noncopy}}
\newcommand{\tread}[6]{#1;#2\vdash #3 \rightsquigarrow #4 \dashv #5;#6}
\newcommand{\trval}[6]{#1;#2\vdash \colorbox{light-gray}{$#3$}~: #4 \rightsquigarrow #5;#6}
\newcommand{\tinstr}[5]{#1;#2\vdash \colorbox{light-gray}{$#3$} \rightsquigarrow #4;#5}
\newcommand{\tfun}[4]{#1;#2|#3\vdash \colorbox{light-gray}{$#4$}}
\newcommand{\envincl}[2]{#1 \Rightarrow #2}
\newcommand{\lookup}[3]{(#1;#2)(#3)}
\newcommand{\venv}{\mathbf{T}}  % Value context
\newcommand{\cenv}{\mathbf{C}}  % Constraints
\newcommand{\lenv}{\Delta}      % Location context
\newcommand{\kenv}{\mathbf{K}}  % Continuation context


%% METAFUNCTIONS
\NewDocumentCommand\envsubst{>{\SplitArgument{1}{;}}m>{\SplitArgument{1}{;}}mm}
  {\@jointwo#1 \Rightarrow^{#3} \@jointwo#2}
\NewDocumentCommand\typsubst{>{\SplitArgument{1}{;}}m>{\SplitArgument{1}{;}}mm}
  {\@jointwo#1 \Rightarrow^{#3} \@jointwo#2}
\NewDocumentCommand\subst{>{\SplitArgument{1}{;}}m}{\mathtt{subst}(\@jointwo#1)}
\NewDocumentCommand\self{>{\SplitArgument{1}{;}}m}{\mathtt{self}(\@jointwo[,]#1)}
\newcommand{\sizeof}[1]{\mathtt{size}(#1)}
\newcommand\extract[1]{\llbracket #1 \rrbracket}

\newcommand{\@where}{\text{\ttfamily\bfseries where}}
\newenvironment{fleqnarray}[2][1em]
  {\begin{fleqn}[#1]\begin{displaymath}\begin{array}{#2}}
  {\end{array}\end{displaymath}\end{fleqn}}
\newenvironment{metafun}[1]{\noindent$\boxed{#1}$\begin{fleqnarray}{lcl}}{\end{fleqnarray}}
\NewExpandableDocumentCommand\where{sm}{
  \multicolumn{3}{l}{\quad\IfBooleanTF{#1}{\phantom{\@where}}{\@where}\;\;#2}
  \\
}
\NewExpandableDocumentCommand\whereblock{t>sm}{
  \IfBooleanT{#1}{\quad\@where\\}
  \multicolumn{3}{l}{
    \arraycolsep=2pt
    \IfBooleanTF{#1}
    {\qquad}
    {\quad\IfBooleanTF{#2}{\phantom{\@where}}{\@where}\hspace{-2pt}\;\;}
    \begin{array}[t]{lcl}#3\end{array}
  }
  \\
}


%% MISC
\renewcommand{\bar}{\widebar}
\renewcommand{\k}{\kappa}
\providecommand{\leftsquigarrow}{%
  \mathrel{\mathpalette\reflect@squig\relax}%
}
\newcommand{\reflect@squig}[2]{%
  \reflectbox{$\m@th#1\rightsquigarrow$}%
}
\DeclareMathSymbol{:}{\mathpunct}{operators}{"3A}

\makeatother

\begin{document}

\section{Syntax}

\begin{displaymath}
  \begin{array}{rrcl}
\textit{path}       & q     & ::= & \epsilon \mid q.n                                           \\
\textit{places}     & \pi   & ::= & x.q                                                         \\
\textit{constants}  & c     & ::= & \<false> \mid \<true> \mid z                                \\
\textit{rvalues}    & R     & ::= & c \mid *\pi \mid *\pi_1 + *\pi_2                            \\
\textit{statements} & S     & ::= & \letvar{x}{\new{n}} \mid \assign{\pi}{R}                    \\
\textit{func body}  & F     & ::= & S; F                                                      \mid
                                    \letcont{k}[\lenv;\cenv]{x}{\venv}{F_1}{F_2} \mid
                                    \jump{k(\bar{x})}                                           \\
                    &       &\mid & \call{f}{\bar{\ell}}{\bar{x}}{k}                          \mid
                                    \ite{*\pi}{F_1}{F_2}                                      \mid
                                    \abort                                                      \\
\textit{base types} & \beta & ::= & \<bool> \mid \<int>                                         \\
\textit{types}      & \tau  & ::= & \fn{\lenv;\cenv;\bar{\own{\ell'}}}{\cenv';\own{\ell_o}}
                                   \mid \r{x:\beta | r} \mid \Pi(\bar{x:\tau})                  \\
\textit{predicates}  & r     & ::= & \dots                                                      \\
\textit{constraints} & \cenv & ::= & \emptyset \mid \cenv, \ell:\tau \mid \cenv, r              \\
\textit{value ctxts} & \venv & ::= & \emptyset \mid \venv, x: \own{\ell}                        \\
\textit{cont. ctxts} & \kenv & ::= & \emptyset \mid \kenv, k: \cont[\lenv;\cenv]{x}{\venv}      \\
\textit{type ctxts}  & \lenv & ::= & \emptyset \mid \lenv,\ell
  \end{array}
\end{displaymath}

\section{Typing}

\begin{mathpar}
  \\
  \textbf{Well-typed rvalues}\hfill \boxed{\trval{\cenv}{\venv}{R}{\tau}{\cenv'}{\venv'}}
  \\
  \infer
  [\textsc{R-add}]
  {
    \lookup{\cenv}{\venv}{\pi_1} = (r_1, \r{\<int> | r'_1}) \\\\
    \lookup{\cenv}{\venv}{\pi_2} = (r_2, \r{\<int> | r'_2})
  }
  {\trval{\cenv}{\venv}{*\pi_1 + *\pi_2}{\r{\<int> | \nu = r_1 + r_2}}{\cenv}{\venv}}

  \infer
  [\textsc{R-copy}]
  {\lookup{\cenv}{\venv}{\pi} = (r,\tau) \\ \iscopy{\tau}}
  {\trval{\cenv}{\venv}{*\pi}{\self{r;\tau}}{\cenv}{\venv}}

  \infer
  [\textsc{R-move}]
  {\lookup{\cenv}{\venv}{\pi} = (r,\tau) \\ \isnoncopy{\tau} \\ n = \text{size}(\tau)}
  {\trval{\cenv}{\venv}{*\pi}{\tau}{(\cenv}{\venv)[\pi\mapsto \uninit_n]}}

  \infer
  [\textsc{R-const}]
  {}
  {\trval{\cenv}{\venv}{c}{\r{\delta(c) | \nu = c}}{\cenv}{\venv}}
\end{mathpar}

\begin{mathpar}
  \\
  \textbf{Well-typed instructions}\hfill \boxed{\tinstr{\cenv}{\venv}{S}{\cenv}{\venv'}}
  \\
  \infer
  [\textsc{S-new}]
  {\cenv'=\cenv*\ell:\uninit_n \\ \venv=\venv*x:\own{\ell}}
  {\tinstr{\cenv}{\venv}{\letvar{x}{\new{n}}}{\cenv'}{\venv'}}

  \infer
  [\textsc{S-assign}]
  {
    \trval{\cenv}{\venv}{R}{\tau'}{\cenv'}{\venv'} \\\\
    \lookup{\cenv}{\venv}{\pi} = (r, \tau) \\ \text{size}(\tau) = \text{size}(\tau')
  }
  {\tinstr{\cenv}{\venv}{\assign{\pi}{R}}{(\cenv'}{\venv')[\pi\mapsto\tau']}}
\end{mathpar}

\begin{mathpar}
  \\
  \textbf{Well-typed functions}\hfill \boxed{\venv|\kenv |- F}
  \\
  \infer
  [\textsc{F-instr}]
  { \tinstr{\cenv}{\venv}{S}{\cenv'}{\venv'} \\
    \tfun{\cenv}{\venv}{\kenv}{F}
  }
  {\tfun{\cenv}{\venv}{\kenv}{S;F}}

  \infer
  [\textsc{F-letcont}]
  {
    \tfun{\cenv'}{\venv',\bar{x:\own{\ell}}}{\kenv,k:\cont[\lenv';\cenv']{x}{\venv'}}{F_1}
    \\\\
    \tfun{\cenv}{\venv}{\kenv,k:\cont[\lenv';\cenv']{x}{\venv'}}{F_2}
  }
  {\tfun{\cenv}{\venv}{\kenv}{\letcont{k}[\lenv';\cenv']{x}{\venv'}{F_1}{F_2}}}

  \infer
  [\textsc{F-jump}]
  {
    \kenv(k) = \cont[\lenv';\cenv']{\venv';\bar{\own{\ell}}} \\\\
    % \envsubst{\cenv;\venv}{\cenv';\venv'*\bar{x:\own{\ell}}}{\theta} \\\\
    \envincl{\cenv;\venv}{\theta(\cenv';\bar{x:\own{\ell}}*\venv')} \\
  }
  {\tfun{\cenv}{\venv}{\kenv}{\jump{k(\bar{x})}}}

  \infer
  [\textsc{F-call}]
  {
    \venv(f)=\fn{\lenv_f;\cenv_f;\bar{\own{\ell_f}}}{\cenv_{o};\own{\ell_{o}}} \\
    \kenv(k) = \cont[\lenv_k;\cenv_k]{\venv_k;\own{\ell_k}} \\\\
    % \\\\
    \theta_1 = [\bar{\ell}/\lenv_f] \\ \theta_2 = [\bar{\ell}/\lenv_k]\cdot[\ell_o/\ell_k] \\
    \envincl{\cenv;\venv}{\theta_1(\cenv_f;\bar{x:\own{\ell_f}})} \\\\
    \envincl{\cenv*\theta_1\cenv_o;\venv - \{\bar{x}\}, y:\own{\ell_o}}{\theta_2(\cenv_k;y:\own{\ell_k}*\venv_k)}
    % \envincl{\cenv*\cenv_o;}
    % \envincl{\cenv,\cenv_o[\bar{\ell}/\lenv_f];\venv}{(\cenv_k;\venv_k)[\bar{\ell}/\lenv_k]}
  }
  {\tfun{\cenv}{\venv}{\kenv}{\call{f}{\bar{\ell}}{\bar{x}}{k}}}

  \infer
  [\textsc{F-if}]
  {
    \lookup{\cenv}{\venv}{\pi} = (r, \r{\<bool>|r'}) \\\\
    \tfun{\cenv,r}{\venv}{\kenv}{F_1} \\
    \tfun{\cenv,\neg r}{\venv}{\kenv}{F_2}
  }
  {\tfun{\cenv}{\venv}{\kenv}{\ite{*\pi}{F_1}{F_2}}}

  \infer
  [\textsc{F-abort}]
  {}
  {\tfun{\cenv}{\venv}{\kenv}{\abort}}
\end{mathpar}

\section{Environment inclusion}

\begin{mathpar}
  \textbf{Environment inclusion}\hfill\boxed{\envincl{\venv_1}{\venv_2}}
  \\
  \infer
  {\text{dom}(\venv_2) \subseteq \text{dom}(\venv_1)}
  {\envincl{\venv_1}{\venv_2}}
\end{mathpar}


% \begin{mathpar}
%   \textbf{Well-formed types and environments}\hfill \boxed{\venv |- \tau \text{ and } |- \venv}
%   \\
%   \infer[Wf-fn]
%   {\venv |- \Pi(\bar{x: \tau}) \\ \venv,\bar{x:\tau} |- \tau'}
%   {\venv |- \<fn>(\bar{x:\tau}) -> \tau'}

%   \infer[Wf-refine]
%   {\lfloor \venv \rfloor,x: \beta |- r \colon \<bool>}
%   {\venv |- \r{x: \beta | r}}

%   \infer[Wf-prod]
%   {\venv, x_0:\tau_0,\dots,x_{i-1}: \tau_{i-1} |- \tau_{i}}
%   {\venv |- \Pi(\bar{x:\tau})}

%   \infer[Wf-empty]
%   {\\}
%   {|- \emptyset}

%   \infer[Wf-binding]
%   {|- \venv\\ \venv |- \tau}
%   {|- \venv, x: \tau}

%   \infer[Wf-pred]
%   {|- \venv\\ \venv |- r \colon \<bool>}
%   {|- \venv, r}
% \end{mathpar}

\section{Subtyping}

\begin{mathpar}
  \textbf{Environment subtyping}\hfill \boxed{\cenv_1 \preceq \cenv_2}
  \\
  \infer[$\preceq$-env-empty]
  {}
  {\cenv_1 \preceq \emptyset}

  \infer[$\preceq$-env-var]
  {\cenv_1 \preceq \cenv_2 \\ \cenv_1 |- \cenv_1(x) \preceq \tau}
  {\cenv_1 \preceq x:\tau, \cenv_2}
\end{mathpar}

\begin{mathpar}
  \textbf{Subtyping}\hfill \boxed{\cenv |- \tau_1 \preceq \tau_2}
  \\
  \infer[$\preceq$-refine]
  {\mathtt{Valid}(\extract{\cenv} \wedge r_1 => r_2[x/y])}
  {\cenv |- \r{x: \beta | r_1} \preceq \r{y: \beta | r_2}}

  \infer[$\preceq$-fun]
  {
    \theta = [\bar{\ell_2}/\bar{\ell_1}]   \\\\
    \cenv*\cenv_2 \preceq \theta\cenv_1                 \\
    \cenv*\cenv_2*\theta\cenv_1' \preceq \cenv_2'[\ell'_1/\ell'_2]
  }
  {\cenv |- \fn{\cenv_1;\bar{\ell_1}}{\cenv_1';\ell'_1} \preceq \fn{\cenv_2;\bar{\ell_2}}{\cenv_2';\ell'_2}}

  \infer[$\preceq$-prod]
  {\cenv*\bar{\ell:\tau} \preceq (\bar{\ell:\tau'})[\bar{\ell}/\bar{y}]}
  {\cenv |- \Pi(\bar{x: \tau}) \preceq \Pi(\bar{y: \tau'})}
\end{mathpar}

\section{Metafunctions}

\begin{metafun}{\cenv_1*\cenv_2=\cenv}
  \cenv_1*\cenv_2,\ell:\tau &=& \cenv_1,\ell:\tau * \cenv_2 \\
  \where{\ell \notin \mathtt{dom}(\cenv_1)}
\end{metafun}

\begin{metafun}{\venv_1*\venv_2=\venv}
  \venv_1*\venv_2,x:\tau &=& \venv_1,x:\tau * \venv_2 \\
  \where{x \notin \mathtt{dom}(\venv_1)}
\end{metafun}

% \begin{metafun}{\tau_1[\pi\mapsto \tau']=\tau_2,n}
%   \tau[\epsilon \mapsto \tau'] &=& \tau'                      \\
%   \uninit_m[\pi.n\mapsto \tau']  &=& \Pi(x_0:\uninit_{n},y: \k )\\
%   \Pi(x_0:\tau_0,\dotsc,x_i:\tau_i,\dotsc,x_m:\tau_m)[n.p \mapsto \tau'] &=& \Pi(x_0:\tau_0,\dotsc,x_i:\tau_i',x_m:\tau_m) \\
%   \where{\sum_{j<i}\text{size}(\tau_j) = n}
%   \where*{\tau'_i = \tau_i[p\mapsto \tau']}
% \end{metafun}

\begin{metafun}{(\cenv;\venv)[\pi\mapsto \tau] = \cenv;\venv}
  (\cenv;\venv)[x.p\mapsto\tau] &=& (\cenv,\ell': \tau'; \venv[x\mapsto \own{\ell'}]) \\
  \whereblock {
    \own{\ell} &=& \venv(x)                    \\
    \tau'      &=& \cenv(\ell)[p\mapsto \tau]
  }
  \where*{\text{fresh}~\ell'}
\end{metafun}

\begin{mathpar}
  \boxed{\envsubst{\cenv_1;\venv_1}{\cenv_2;\venv_2}{\theta}}\hfill\,
  \\

  \infer{}{\envsubst{\cenv_1;\venv_1}{\cenv_2;\emptyset}{\emptyset}}

  \infer
  {
    \envsubst{\cenv_1;\venv_1}{\cenv_2; \venv_2}{\theta_1}   \\
    \typsubst{\cenv_1;\venv_1(x)}{\cenv_2; \own{\ell_2}}{\theta_2}
  }
  {\envsubst{\cenv_1;\venv_1}{\cenv_2;\venv_2,x:\own{\ell_2}}{\theta_1\cdot\theta_2}}
\end{mathpar}

% \begin{metafun}{\subst{\cenv_1,\venv_1;\cenv_2,\venv_2} = \theta}
%   \subst{\cenv_1,\venv_1;\cenv_2,\venv_2,x:\own{\ell_2}} &=& \theta_1 \cdot \theta_2 \cdot [\ell_1/\ell_2] \\
%   \whereblock{
%     \own{\ell_1} &=& \venv_1(x)                                \\
%     \theta_2     &=& \subst{\venv_1; \venv_2}                  \\
%     \theta_2     &=& \subst{\cenv_1(\ell_1);\cenv_2(\ell_2)}
%   }
% \end{metafun}

\begin{mathpar}
  \boxed{\typsubst{\cenv_1;\tau_1}{\cenv_2;\tau_2}{\theta}}\hfill\,
  \\

  \infer
  {\typsubst{\cenv_1;\cenv_1(\ell_1)}{\cenv_2;\cenv_2(\ell_2)}{\theta}}
  {\typsubst{\cenv_1;\own{\ell_1}}{\cenv_2;\own{\ell_2}}{\theta\cdot[\ell_1/\ell_2]}}

  \infer
  {\tau_1 \neq \own{\ell_1} \vee \tau_2 \neq \own{\ell_2}}
  {\typsubst{\cenv_1;\tau_1}{\cenv_2;\tau_2}{\emptyset}}
\end{mathpar}

\begin{metafun}{\subst{\tau_1;\tau_2} = \theta}
  \subst{\own{\ell_1};\own{\ell_2}} &=& [\ell_1/\ell_2] \\
  \subst{\tau_1; \tau_2}            &=& \emptyset
\end{metafun}

\begin{metafun}{\tau.q = \tau'}
  \tau.\epsilon &=& \tau \\
  \Pi(x_0:\tau_0,\dotsc,x_n: \tau_n,x_m: \tau_m).n.q &=& \tau_n.q \\
  % \where{\sum_{j<i}\text{size}(\tau_j) = n}
\end{metafun}

\begin{metafun}{\lookup{\cenv}{\venv}{\pi} = (r, \tau)}
  \lookup{\cenv}{\venv}{x.q} &=& (\ell.q, \tau.q) \\
  \whereblock{
    \own{\ell} &=& \venv(x) \\
    \tau       &=& \cenv(y)
  }
\end{metafun}

\begin{metafun}{\mathtt{self}(p, \tau) = \tau'}
  \mathtt{self}(p,\r{x:\beta | r}) &=& \r{x:\beta | x = p} \\
  \mathtt{self}(p, \tau)           &=& \tau
\end{metafun}

\begin{metafun}{\extract{\cenv} = r}
  \extract{\emptyset       } &=& \<true> \\
  \extract{\cenv,r        } &=& \extract{\cenv} \wedge r                     \\
  \extract{\cenv,\ell:\tau} &=& \extract{\cenv} \wedge \extract{\tau}_{\ell} \\
\end{metafun}

\begin{metafun}{\extract{\tau}_{\ell.q} = r}
  \extract{\r{y: \beta | r}}_{\ell.q}      &=& r[\ell.q/y]                                       \\
  \extract{\Pi(\bar{x_i:\tau_i})}_{\ell.q} &=& \theta (\bigwedge_i \extract{\tau_i}_{\ell.q.i}) \\
  \where{\theta = [\bar{\ell.q.i}/\bar{x_i}]}
  \extract{\tau}_{\ell.q}                  &=& \<true>                                           \\
\end{metafun}

\newpage

\section{Examples}

\subsection{Tracking variable versions}
\begin{minted}{rust}
fn ris() {
  let mut p = (1, 2);
  let x = p.0;
  p.0 = 3;
  let y = p.0;
  let z = p.1;
  // What can we say about x, y and z?
}
\end{minted}

\begin{lambdalr}
fn ris(;;) ret k($r_0: ()$; $r_0$) =
  let p = new(2) in   // $p_0: \lightning_2; p: \own{p_0}$
  p.0 := 1;           // $\dots, p_1: \r{i32 | \nu = 1} \times \lightning_1; p: \own{p_1}$
  p.1 := 2;           // $\dots, p_2: \r{i32 | \nu = 1} \times \r{i32 | \nu = 2}; p: \own{p_2}$
  let x = new(1) in   // $\dots, x_0: \lightning_1; p: \own{p_2}, x: \own{x_0}$
  x := *p.0;          // $\dots, x_1: \r{i32 | \nu = p_2.0}; p: \own{p_2}, x: \own{x_1}$
  p.0 := 3;           // $\dots, p_3: \r{i32 | \nu = 3} \times \r{i32 | \nu = 2}; p: \own{p_3}, x:\own{x_1}$
  let y = new(1) in   // $\dots, y_0: \lightning_1; p: \own{p_3}, x:\own{x_1}, y: \own{y_0}$
  y := *p.0;          // $\dots, y_0: \r{i32 | \nu = p_3.0}; p: \own{p_3}, x:\own{x_1}, y: \own{y_1}$
  let z = new(1) in   // $\dots, z_0: \lightning_1; p: \own{p_3}, x:\own{x_1}, y:\own{y_1}, z:\own{z_0}$
  z := *p.1;          // $\dots, z_1: \r{i32 | \nu = p_3.1}; p:\own{p_3}, x:\own{x_1}, y:\own{y_1}, z:\own{z_1}$
  jump(())
\end{lambdalr}

\subsection{Basic control flow}

\begin{minted}{rust}
fn abs(mut x: i32) -> i32 {
  if (x < 0)  {
    x = -x;
  }
  x
}
\end{minted}

\begin{lambdalr}
fn abs($x_0: i32$; x: own($x_0$)) ret k($r_0:\r{i32|\nu > 0}$; r: own($r_0$)) =
  // $x_0: i32$; $x: \own{x_0}$
  let b: bool = *x < 0 in  // $\dots, b_0:\r{bool | \nu \Leftrightarrow x_0 < 0}$; $b:\own{b_0}$
  if b then                // $\dots, b$
    x := 0 - *x            // $\dots, x_1: \r{i32 | \nu = 0 - x_0}$; $x: \own{x_1}$
    jump k(x)
      // $x_0: i32, b_0:\r{i32 | \nu \Leftrightarrow x_0 < 0}, b, x_1:\r{i32 | \nu = 0-x_0}$; $x:\own{x_1}, b:\own{b_0}$
      // $\preceq$
      // $m_0:\r{i32 | \nu > 0}$; $m:\own{m_0}$

  else  // $\dots, \neg b$
    jump k(x)
      // $x_0: i32, b_0:\r{i32 | \nu \Leftrightarrow x_0 < 0}, \neg b$; $x:\own{x_1}, b:\own{b_0}$
      // $\preceq$
      // $r_0:\r{i32 | \nu > 0}$; $x:\own{r_0}$
\end{lambdalr}


\clearpage
\subsection{Dependent function}

\begin{minted}[escapeinside=\%\%]{rust}
fn ira(a: i32, b: {i32 | b > a}) -> {i32 | v > 0} {
  b - a
}
\end{minted}

\begin{lambdalr}
fn ira($a_0: i32$,$b_0:\r{i32|\nu>a_0}$; a:own($a_0$), b:own($b_0$)) ret k($r_0:\r{i32 | \nu>0}$; own($r_0$)) =
  // $a_0: i32,b_0:\r{i32 | \nu > a_0}$; $a : \own{a_0}$, $b : \own{b_0}$
  let t = new(1);  // $\dots,t_0: \lightning_1$; $\dots,t$ $:$ own($b_0$)
  t := *a - *b;    // $\dots,t_1: \r{i32 | \nu = a_0 - b_0}$; $\dots,t : \own{t_1}$
  jump k(t) // $a_0: i32, b_0: \r{i32 | \nu > a_0}, t_0: \lightning_1,t_1: \r{i32 | \nu = a_0 - b_0}$;$a: \own{a_0}, b: \own{b_0}, t: \own{t_1}$
            // $\preceq$
            // $r_0:\r{i32 | \nu > 0}; t:\own{r_0}$
\end{lambdalr}

\subsection{Function call}

\begin{minted}{rust}
fn f(y: {i32 | y >= 0}) -> {i32 | v >= y};

fn count_zeros(n: {i32 | n >= 0}) -> {v: i32 | v >= 0} {
  let mut i = 0;
  let mut c = 0;
  while (i < n) {
    if (f(i) == 0) {
      c += 1;
    }
    i += 1;
  }
  c
}
\end{minted}

\begin{lambdalr}
fn count_zeros($n_0: \r{i32|\nu \geq 0}$; n: own($n_0$)) ret k($r_0: \r{i32 | \nu \geq 0}$;own($r_0$)) =
  letcont b0($i_1: \r{i32| \nu \geq 0}, c_1: \r{i32 | c \geq 0}$; n: own($n_0$), i: own($i_1$); c: own($c_1$); ) =
    let t1: bool = *i < *n;
    if *t1 then
      letcont b1($x_0: i32$; $n:\own{n_0}, i:\own{i_1}, c:\own{c_1}$; x:$\own{x_0}$) =
        letcont b2(; $n:\own{n_0}, i:\own{i_1}, c:\own{c_1}$; ) =
          i := *i + 1;
          jump b0()
        in
        let t2: bool = *x == 0;
        if *t2 then
          c := *c + 1;
          jump b2()
        else
          jump b2()
      in
      call f(i) ret b1
        // $n_0:\r{i32|\nu\geq 0}, i_1:\r{i32|\nu\geq 0}, c_1:\r{i32|c\geq 0},t1_0:\r{bool|i_1<n_0},t1_0$
        // ;$n:\own{n_0},i:\own{i_1},c:\own{c_1}$
        // $\preceq$
        // $y_0: \r{i32 | \nu\geq 0}$; $i:\own{y_0}$
        //
        // $n_0:\r{i32|\nu\geq 0}, i_1:\r{i32|\nu\geq 0}, c_1:\r{i32|c\geq 0},t1_0:\r{bool|i_1<n_0},t1_0,r_0:\r{i32|\nu\geq y_0[y_0/i_1]}$;
        // $n:\own{n_0},i:\own{i_1},c:\own{c_1}, r:\own{r_0}$
        // $\preceq$
        // $x_0: i32$; $n: \own{n_0}, i:\own{i_1}, c:\own{c_1}, r:\own{x_0}$
    else
      jump k(c1)
  in
  let i: int = 0;
  let c: int = 0;
  jump b0()
\end{lambdalr}

\subsection{The sum example with references}
\begin{minted}{rust}
fn sum(n: i32) -> {i32 | v >= n} {
  let mut i = 0;
  let mut r = 0;
  while (i < n) {
    i += 1;
    r += i;
  }
  r
}
\end{minted}

\begin{lambdalr}
fn sum($n_0: i32$; $n$: own($n_0$)) ret k($m_0: \r{i32 | \nu \geq n_0}$; m: own($m_0$)) =
  // $n_0: i32$; $n: \own{n_0}$
  let i = new(1);
  i := 0;           // $\dots, i_0:\r{i32 | \nu = 0}$; $\dots, i:\own{i_0}$
  let r = new(1);
  r := 0;           // $\dots, r_0:\r{i32 | \nu = 0}$; $\dots, r:\own{r_0}$
  letcont loop($i_1: i32, r_1: \r{i32 | \nu \geq i_1}$; n: own($n_0$), i: own($i_1$), r: own($r_1$)) =
    // $\dots, i_1: i32, r_1:\r{i32 | \nu \geq i_1}$; $n: \own{n_0}, i: \own{i_1}, r: \own{r_1}$
    let b = new(1);
    b := *i < *n;   // $\dots, b_0:\r{i32|\nu \Leftrightarrow i_1 > n_0}$; $b: \own{b_0}$
    if b then
      i := *i + 1;  // $\dots, i_2:\r{i32 | \nu = i_1 + 1}$; $i: \own{i_2}$
      r := *i + *r; // $\dots, r_2:\r{i32 | \nu = i_2 + r_1}$; $r: \own{r_2}$
      jump loop()
        // $n_0: i32, i_0:\r{i32 | \nu = 0},r_0:\r{i32 | \nu=0}, i_1: i32, r_1:\r{i32 | \nu \geq i_1},b_0:\r{i32|\nu \Leftrightarrow i_1 > n_0},$
        // $b, i_2:\r{i32 | \nu = i_1 + 1}, r_2:\r{i32 | \nu = i_2 + r_1}$;
        // $n:\own{n_0}, i:\own{i_2}, r:\own{r_2}$
        // $\preceq$
        // $i_1: i32, r_1 : \r{i32 | \nu \geq i_1}$;
        // $n: \own{n_0}, i:\own{i_1}, r:\own{r_1}$
    else
      jump k(r)
  jump loop()
\end{lambdalr}

\clearpage
\subsection{Non-copy type}
\begin{minted}{rust}
// Types are by default non-copy
struct Point {
  x: i32,
  y: i32
}

fn hipa(r: (Point, Point)) -> i32 {
  let a = r.0;
  a.x + r.1.y
}
\end{minted}

\begin{lambdalr}
fn hipa($r_0: \verb|Point|\times\verb|Point|$, r: own($r_0$)) ret k($m_0: i32$; m: own($m_0$))
  // $r_0: \verb|Point|\times\verb|Point|$; $r: \own{r_0}$
  let a: Point = *r.0 in  // $\dots, r_1:\lightning_2\times \verb|Point|$; $r: \own{r_1}, a:\own{r_0.0}$
  let b: i32 = *a.0 in    // $\dots, b_0:\r{i32 | \nu = r_0.0}$; $\dots, b: \own{b_0}$
  let c: i32 = *r.1.1 in  // $\dots, c_0:\r{i32 | \nu = r_1.1.1}$; $\dots,c: \own{c_0}$
  let d: i32 = *b + *d in // $\dots, d_0:\r{i32 | \nu = b_0 + d_0}$; $\dots, d: \own{d_0}$
  jump k(d)
\end{lambdalr}
\end{document}
